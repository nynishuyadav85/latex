\documentclass[]{article}
\pagestyle{empty}
\usepackage{amsmath}

\begin{document}

Hello, This my first \LaTeX\ document!

The rectangle is of length $(x+2)$ and $(x+3)$.
The Equation $${A(x)=x^2+4x+3}$$ gives the area of rectangle

-----------------------------------------------------------------------------------------

\textbf{Common mathmatical Notation}

SuperScript $$2x^3$$
$$2x^{34}$$
$$2x^{2x+4}$$
$$2x^{3x^{54}}$$

SubScripts  $${x_1}$$
$${x_{12}}$$
$${x_{1_{2_{3_4}}}}$$
$${a_1}, {a_2}, \ldots {a_{100}}$$

Greek Letters   

$$\pi$$
$$\Pi$$
$$\alpha$$
$$\aleph$$
$$A= \pi r^2$$

Trignometry Function

$$y=\sin x$$
$$ y = \cos x $$
$$ y = \csc \theta $$
$$ y = \sin ^{-1} x $$
$$ y = \arcsin x $$


Log Function

$$ y = \log x $$
$$ y = \log_5 x $$
$$ y = \ln x $$

Roots

$$ \sqrt{2} $$
$$ \sqrt[3]{2} $$
$$ \sqrt{x^2 + y^2} $$
$$ \sqrt{ 1 + \sqrt{x} } $$


Fraction \\[16pt]

About $ \frac{2}{3} $ of glass is full.\\[16pt]
About $ \displaystyle \frac{2}{3} $ of glass is full.\\[16pt]
About $  \dfrac{2}{3} $ of glass is full.

$$ \frac{ \sqrt{x+1} } { \sqrt{x+2} } $$
$$ \frac{ 1 } { 1+ \frac{1}{2} } $$

\end{document}