\documentclass[]{article}
\pagestyle{empty}
\usepackage{amsmath, amsfonts, amssymb}
\usepackage{float}
\usepackage{enumerate}
\usepackage{hyperref}
\usepackage{tikz,pgfplots}
\usepackage{graphicx}

\parindent 0px

\title{\LaTeX\ Document Learning}
\author{Nishant Yadav}
\date{\today}

\def\eq1{y=\dfrac{x}{3^3+x-1}}
\newcommand{\set}[1] {\setlength\itemsep{#1 em}}
\newcommand\calculator{\tikz{
		\node (c) [inner sep=0pt, draw, fill=black, anchor=south west]{\phantom{N}};
		\begin{scope}[x=(c.south east),y=(c.north west)]    \fill[white] (.1,.7) rectangle (.9,.9);    
		\foreach \x in {.1, .33, .55, .79}{    
		\foreach \y in {.1, .24, .38, .53}{    
		\fill[white] (\x,\y) rectangle +(.11,.07);}} 
		\end{scope} }}
		\def\calcicon#1{\noindent#1 \calculator\ }


\begin{document}
\tableofcontents
\maketitle

Hello, This my first \LaTeX\ document!

The rectangle is of length $(x+2)$ and $(x+3)$.
The Equation $${A(x)=x^2+4x+3}$$ gives the area of rectangle

-----------------------------------------------------------------------------------------

\textbf{Common mathmatical Notation}

SuperScript $$2x^3$$
$$2x^{34}$$
$$2x^{2x+4}$$
$$2x^{3x^{54}}$$

SubScripts  $${x_1}$$
$${x_{12}}$$
$${x_{1_{2_{3_4}}}}$$
$${a_1}, {a_2}, \ldots {a_{100}}$$

Greek Letters   

$$\pi$$
$$\Pi$$
$$\alpha$$
$$\aleph$$
$$A= \pi r^2$$

Trignometry Function

$$y=\sin x$$
$$ y = \cos x $$
$$ y = \csc \theta $$
$$ y = \sin ^{-1} x $$
$$ y = \arcsin x $$


Log Function

$$ y = \log x $$
$$ y = \log_5 x $$
$$ y = \ln x $$

Roots

$$ \sqrt{2} $$
$$ \sqrt[3]{2} $$
$$ \sqrt{x^2 + y^2} $$
$$ \sqrt{ 1 + \sqrt{x} } $$



Fraction \\[16pt]

About $ \frac{2}{3} $ of glass is full.\\[16pt]
About $ \displaystyle \frac{2}{3} $ of glass is full.\\[16pt]
About $  \dfrac{2}{3} $ of glass is full.

$$ \frac{ \sqrt{x+1} } { \sqrt{x+2} } $$
$$ \frac{ 1 } { 1+ \frac{1}{4} } $$

Brackets \\[16pt]

States that $a(b+c)=ab+ac$, for all $ a, b, c \in \mathbb{R} $ \\[6pt]
Square $a$ , $[a]$ \\[6pt]
Curly Bracket ${A}$ , $\{working\}$ \\[6pt]
Doller Sign $\$$ \\[6pt]
$$ 2 \left ( \frac{2}{1^{2-1}} \right ) $$
$$ 2 \left [ \frac{2}{1^{2-1}} \right ] $$
$$ 2 \left \{ \frac{2}{1^{2-1}} \right \} $$
$$  2 \left \langle \frac{2}{1^{2-1}} \right \rangle  $$
$$  2 \left | \frac{2}{1^{2-1}} \right |  $$ 
$$ \left. \frac{dy}{dx} \right| _{x=1} $$
$$ \left( \frac{1}{1 + \left( \frac{1}{1+x} \right)} \right)  $$

Tables \\[6pt]

\begin{tabular}{|c|c|c|c|c|c|}
    \hline

    $ x $ & 1 & 2 & 3 & 4 & 5 \\ \hline
    $ f(x) $ & 10 & 11 & 12 & 14 & 15  \\ \hline


\end{tabular}

\vspace{2cm}

\begin{table}[H]
    
    \centering
    \def\arraystretch{1.5}

\begin{tabular}{|c|c|c|c|c|c|}
    \hline

    $ x $ & 1 & 2 & 3 & 4 & 5 \\ \hline
    $ f(x) $ & $ \frac{1}{2} $ & 11 & 12 & 14 & 15  \\ \hline


\end{tabular}

\caption{these value $ f(x) $}

\end{table}

Arrays \\[6pt]

\begin{align}
5x^2-9=x+3 \\
5x^2-9=x+3 
\end{align}

\begin{align*}
    5x^2-9=x+3 \\
    5x^2-9=x+3 
    \end{align*}

    \begin{align}
        5x^2-9=x+3 \\
        5x^2-9=x+3 
        \end{align}

 Lists \\[6pt]

 \begin{enumerate}
\item nishu
\item deepak
\item neelam
   \begin{enumerate}
    \item gla
    \item bank
   \end{enumerate}
\item last   
    
 \end{enumerate}

 \vspace{2cm}


 \begin{enumerate}[A.]
    \item nishu
    \item deepak
    \item neelam
 \end{enumerate}

 \vspace{2cm}


 \begin{enumerate}
    \item[] nishu
    \item[] deepak
    \item[] neelam
 \end{enumerate}


 \vspace{2cm}

 \begin{itemize}
    \item nishu
    \item deepak
    \item neelam
       \begin{itemize}
        \item gla
        \item bank
       \end{itemize}
    \item last   
        
     \end{itemize}



 Text and Document Formatting \\[6pt]

 this is a \textit{test}

 this is a \textbf{test}

 this is a \textsc{test}

 this is a \texttt{test}

 please visit \href{www.google.com}{google}

\vspace{2cm}


i am nishant yadav

i am \begin{large} nishant yadav \end{large}

i am \begin{Large} nishant yadav \end{Large}

i am \begin{huge} nishant yadav \end{huge}

i am \begin{Huge} nishant yadav \end{Huge}

i am nishant yadav

i am \begin{normalsize} nishant yadav \end{normalsize}

i am \begin{small} nishant yadav \end{small}

i am \begin{scriptsize} nishant yadav \end{scriptsize}

i am \begin{tiny} nishant yadav \end{tiny}

\vspace{2cm}

\begin{center} this is a center \end{center}

\begin{flushleft} this is a left \end{flushleft}

\begin{flushright} this is a right \end{flushright}

\vspace{2cm}


\section{Lists}
 \subsection{learning}
 \subsection{thinking}
\section{formatting}
 \subsection{text}
 \subsection{words} 


 Macros \\[6pt]


 \textbf{Critical Thinking Questions}


\begin{figure}[H]
\centering
\caption{The Squeeze Theorem}
\end{figure}




\begin{enumerate}
\set{1.2}
\item \calculator\ Let's examine the function $\eq1$.
\item This is the symbol for the set of all real numbers: $\mathbb{R}$.
\item This is the symbol for the set of integers: $\mathbb{Z}$.
\item This is the symbol for the set of rationals: $\mathbb{Q}$.
\item Is it possible for a sequence to converge to two different numbers? If so, give an example. If not, explain why not.
\item Explain how to use partial sums to determine if a series converges or diverges. Give an example
\item Explain why $\int\limits_{1}^{\infty} f(x)\,dx$ and $\sum\limits_{n=1}^{\infty} a_n$ need not converge to the same value, even if they are both convergent.
\item  In your own words, explain the Alternating Series Remainder Theorem. How is this theorem useful?
\item Explain the difference between absolute and conditional convergence. Give an example of each.
\item The Ratio Test is inconclusive if $\displaystyle{\lim\limits_{n \to \infty} \left| \frac{a_{n+1}}{a_n} \right| =1}$. Give an example of one convergent series and one divergent series for which $\displaystyle{\lim\limits_{n \to \infty} \left| \frac{a_{n+1}}{a_n} \right| =1}$. Explain how you determined your examples.
\end{enumerate}




\end{document}